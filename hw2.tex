\documentclass{homework}
\author{Morales, Samantha}  % Uncomment with your name
\class{MATH 3503: Tashfeen's Discrete Mathematics}
\date{\today}  % Uncomment with your semester

\title{Homework 2}
\address{
  Computer Science, %
  Petree College of Arts \& Sciences, %
  Oklahoma City University
}

\begin{document} \maketitle

\question Please read chapter 2 of Chartrand et al. and write a couple sentences about a topic/example/concept that you found difficult or interesting and why?

% solution goes here
\begin{sol}
    The distinction between subsets and proper subsets was very interesting to me because it seems almost counter-intuitive.
    usually we expect comparison operators that have an equality clause to be more restrictive than their non-equality counterparts, (e.x. $\leq$ \& $\geq$). 
    However, subsets $\subseteq$ appear to be less restrictive than proper subsets $\subset$.
\end{sol}

\question How many elements are in $\mathcal{P}(A)$ if $A = \{n \in \Z: |n| \leq 5\}$?

% solution goes here
\begin{sol}
    $|A|=11$ \\
    $|\mathcal{P}(A)|=2^{11}$
\end{sol}

\question Let $A = \{0, \{0\}, \{0, \{0\}\}\}$.
\begin{enumerate}[label=(\alph*)]
    \item Determine which of the following are elements of $A: 0, \{0\}, \{\{0\}\}$.

    % solution goes here
    \begin{sol}
        $0, \{0\} \in A$
    \end{sol}

    \item Determine $|A|$.

    % solution goes here
    \begin{sol}
        $|A| = 3$
    \end{sol}
    
    \item Determine which of the following are subsets of $A: 0, \{0\}, \{\{0\}\}$.

    % solution goes here
    \begin{sol}
        $\{0\} \subset A$ \smallbreak
        $\{\{0\}\} \subset A$

    \end{sol}
    
    For (d)-(i), determine the indicated sets.
    \item $\{0\} \cap A$. 

    % solution goes here
    \begin{sol}
        $0$
    \end{sol}
    
    \item $\{\{0\}\} \cap A$.

    % solution goes here
    \begin{sol}
        $\{0\}$
    \end{sol}
    
    \item $\{\{\{0\}\}\} \cap A$.

    % solution goes here
    \begin{sol}
        $\nil$
    \end{sol}
    
    \item $\{0\} \cup A$.

    % solution goes here
    \begin{sol}
        $\{0, \{0\},\{0,\{0\}\}\}$
    \end{sol}
    
    \item $\{\{0\}\} \cup A$.

    % solution goes here
    \begin{sol}
        $\{0, \{0\},\{0,\{0\}\}\}$
    \end{sol}
    
    \item $\{\{\{0\}\}\} \cup A$.

    % solution goes here
    \begin{sol}
        $\{0, \{0\},\{0,\{0\}\}, \{\{0\}\}\}$
    \end{sol}
\end{enumerate}
\question For two sets $A$ and $B$ of real numbers, the set $A \cdot B$ is defined by,
\[
    A \cdot B = \{ab: a \in A, b \in B\}.
\]
Determine each of the following sets.
\begin{enumerate}
    \item $A \cdot B$ for $A = \{\frac{1}{2}, 1, \sqrt{2}\}$ and $B = \{\sqrt{2}, 2, 4\}$.

    % solution goes here
    \begin{sol}
        \[
            \{\frac{\sqrt{2}}{2}, 1, 2, \sqrt{2}, 2, 4, 2, 2\sqrt{2},4\sqrt{2}\}
        \]
    \end{sol}
    
    \item $\R \cdot \R$.

    % solution goes here
    \begin{sol}
        $\R$
    \end{sol}
    
    \item $\R \cdot C$ where $C \subseteq \R$ with $|C| = 2$.

    % solution goes here
    \begin{sol}
        $\R$
    \end{sol}
\end{enumerate}

\question For $A = \{1, 2\}, B = \{-1, 0, 1\}$ and the universal set $U = \{-2, -1, 0, 1, 2\}$, determine
\begin{enumerate}[label=(\alph*)]
    \item $A \cup B$.

    % solution goes here
    \begin{sol}
        $\{1, 2, -1, 0\}$
    \end{sol}
    
    \item $A \cap B$.

    % solution goes here
    \begin{sol}
        $\{1\}$
    \end{sol}
    
    \item $A - B$.

    % solution goes here
    \begin{sol}
        $\{2\}$
    \end{sol}
    
    \item $\overline{B}$.

    % solution goes here
    \begin{sol}
        $\{-2,2\}$
    \end{sol}
    
    \item $A \times B$.

    % solution goes here
    \begin{sol}
        $\{(1,-1),(1,0),(1,1),(2,-1),(2,0),(2,1)\}$    
    \end{sol}
    
\end{enumerate}

\question Give examples of three sets $A, B$ and $C$ such that
\begin{enumerate}[label=(\alph*)]
    \item $A \subseteq B \not\subset C$.

    % solution goes here
    \begin{sol}
        $A=\{1,2,3\}$
        $B=\{1,2,3\}$
        $C=\{0\}$
    \end{sol}

    
    \item $A \subseteq B, B \in C$ and $A \cap C = \nil$.

    % solution goes here
    \begin{sol}
        $A=\{1\}$
        $B=\{1,2,3\}$
        $C=\{\{1,2,3\},3\}$
    \end{sol}
    
    \item $A \in B, A \subset B$ and $A \not\subseteq C$.

    % solution goes here
    \begin{sol}
        $A=\{3\}$
        $B=\{\{3\},3\}$
        $C=\{2\}$
    \end{sol}
    
    \item $A \in B, A \not\subseteq B$ and $B \in C$.

    % solution goes here
    \begin{sol}
        $A=\{3\}$
        $B=\{\{3\}\}$
        $C=\{\{\{3\}\}\}$
    \end{sol}
\end{enumerate}
\end{document}
