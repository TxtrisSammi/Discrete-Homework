\documentclass{homework}
\author{Morales}  % Uncomment with your name
\class{MATH 3503: Tashfeen's Discrete Mathematics}
\date{Spring 2023}  % Uncomment with your semester
\title{Homework 1}
\address{
  Computer Science, %
  Petree College of Arts \& Sciences, %
  Oklahoma City University
}

\begin{document} \maketitle

\question Please read chapters 0 and 1 of Chartrand et al. and write a couple sentences about a  
                              /example/concept that you found difficult or interesting and why?

\question\label{a} Let $P, Q$ and $R$ be statements. Determine whether the following is true.
\[
    P \oplus (Q \oplus R) \is (P \oplus Q) \oplus R. \quad \text{Use a truth table.}
\]

  \begin{sol}
\tbl<tb1>{Truth table for question \ref{a}.} {
  $P$ & $Q$ & $R$ & $P \oplus (Q \oplus R)$ & $(P \oplus Q) \oplus R$ \\
  T   & T   & T   &  T                       &  T                       \\
  T   & T   & F   &  F                       &  F                       \\
  T   & F   & T   &  F                       &  F                       \\
  T   & F   & F   &  T                       &  T                       \\
  F   & T   & T   &  F                       &  F                       \\
  F   & T   & F   &  T                       &  T                       \\
  F   & F   & T   &  T                       &  T                       \\
  F   & F   & F   &  F                       &  F                       \\
}
  Hence, $P \oplus (Q \oplus R) \is (P \oplus Q) \oplus R$ is a true statement.
\end{sol}


\question\label{b} Let $P, Q$ and $R$ be statements. Determine whether the following is true.
\[
    P \vee (Q \oplus R) \is (P \vee Q) \oplus (P \vee R). \quad \text{Use a truth table.}
\]

% type your solution here
 \begin{sol}
\tbl<tb2>{Truth table for question \ref{b}.} {
  $P$ & $Q$ & $R$ & $P \vee (Q \oplus R)$ & $(P \vee Q) \oplus (P \vee R)$ \\
  T   & T   & T   & T                       & F                        \\
  T   & T   & F   & T                       &                         \\
  T   & F   & T   & T                       &                         \\
  T   & F   & F   & T                       &                         \\
  F   & T   & T   & F                       &                         \\
  F   & T   & F   & T                       &                         \\
  F   & F   & T   & T                       &                         \\
  F   & F   & F   & F                       &                         \\
 }
  Hence, $P \vee (Q \oplus R) \is (P \vee Q) \oplus (P \vee R)$ is a false statement.
\end{sol}
 




\question For an integer $n$, consider the open sentences
\[
    P(n): n(n + 1)(2n + 1)/6\text{ is even.} \quad Q(n): (n + 1)^2 (n + 2)^2 /4\text{ is even.}
\]
Determine whether the biconditionals $P(1) \iff Q(1)$ and $P(2) \iff Q(2)$ are true or false.

% type your solution here
\begin{sol}
  The statments $P(1)$ and $Q(1)$ are both false \\
  hence, the biconditional $P(1) \iff Q(1)$ is true. \bigskip

  The statment $P(2)$ is false while the statement $Q(2)$ is true \\
  hence, the biconditional $P(2) \iff Q(2)$ is false.
\end{sol}

\question\label{c} For statements $P$ and $Q$, determine whether the compound statement
\[
    (P \wedge (\sim Q)) \Ra (P \vee Q)
\]
is a tautology, a contradiction or neither.

% type your solution here
\begin{sol}
  \tbl<tb3>{Truth table for question \ref{c}.} {
  $P$ & $Q$  & $(P \wedge (\sim Q))$ & $(P \vee Q)$ & $(P \wedge (\sim Q)) \Ra (P \vee Q)$  \\
  T   & T      & F                       & T        &         T       \\
  T   & F      & T                       & T        &         T       \\
  F   & T      & F                       & T        &         T       \\
  F   & F      & F                       & F        &          T      \\
 }
  The compound statement $(P \wedge (\sim Q)) \Ra (P \vee Q)$ is a tautology.
\end{sol}

\question Each of the following statements is an implication $P \Ra Q$. For each statement, indicate what $P$ and $Q$ are.
\begin{enumerate}[label=(\alph*)]
    \item I'm going to my class reunion only if I lose weight.

          % type your solution here
          \begin{sol}
            "I'm going to my class reunion" is the $P$ statement and "I lose weight" is the $Q$ statement.
          \end{sol}

    \item To win a free \$20 gift certificate, I must spend \$100 at the store.

          % type your solution here
          \begin{sol}
            "To win a free \$20 gift certificate" is the $P$ statement and "I must spend \$100 at the store" is the $Q$ statement.
          \end{sol}

    \item To win the game, it is necessary that we score a touchdown.

          % type your solution here
          \begin{sol}
            "To win the game" is the $P$ statement and "it is necessary that we score a touchdown" is the $Q$ statement.
          \end{sol}

    \item It is necessary to do research to be promoted to professor.

          % type your solution here
          \begin{sol}
            "To be promoted to professor" is the $P$ statement and "It is necessary to do research" is the $Q$ statement.
          \end{sol}

    \item I'll get an A on this exam if I'm lucky.

          % type your solution here
          \begin{sol}
            "I'll get an A on this exam" is the $P$ statement and "if I'm lucky" is the $Q$ statement.
          \end{sol}

    \item All I need is a B on the final exam to get an A in the course. 

          % type your solution here
          \begin{sol}
            "All I need is a B on the final exam" is the $P$ statement and "to get an A in the course" is the $Q$ statements.
          \end{sol}

\end{enumerate}

\end{document}
