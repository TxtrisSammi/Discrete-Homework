\documentclass{homework}
\author{Morales, Samantha}  % Uncomment with your name
\class{MATH 3503: Tashfeen's Discrete Mathematics}
\date{\today}  % Uncomment with your semester
\title{Homework 6}
\address{
  Computer Science, %
  Petree College of Arts \& Sciences, %
  Oklahoma City University
}

\begin{document} \maketitle

\textit{Please give your final answer as a number where possible as well as show work for full credit.}

\question Please read chapter 9 of Chartrand et al. and write a couple sentences about a topic/example/concept that you found difficult or interesting and why?

\begin{sol}
    the stars and bars was really intersting, it seems to work for quite a lot of solutions as long as you adjust accordingly which can make it really useful like in question 6
\end{sol}

\question{\label{q16}} What is the exact term in $\paren{\sqrt{x} - \frac{1}{\sqrt{x}}}^{10}$ containing $x$?

\begin{sol}
    $\sum_{k=0}^{10} \binom{10}{k} \sqrt{x}^{10-k}(-\frac{1}{\sqrt{x}})^k = \sum_{k=0}^{10}\binom{10}{k}(-1)^kx^{(0.5)10-2k}$ \\
    $x^{5-k} = x^1, 5-k=1, k=4$ \\
    $\binom{10}{4}(-1)^4x^{(0.5)(10-2(4))} = \binom{10}{4}(1)x^{0.5(2)} = 210x$
\end{sol}
\question Use the Binomial Theorem to simplify the sum,
\[
    2^{n^2} + \binom{n}{1}2^{n(n-1)} + \binom{n}{2}2^{n(n-2)} + \cdots + \binom{n}{n-1}2^n + 1
\]

\begin{sol}
    $\sum_{k=0}^{n}\binom{n}{k}(2^n)^{n-k}(1)^k = (2^n + 1)$
\end{sol}

\question How many different 8-digit numbers can be obtained by permuting the digits in the number $70,440,704$? (Since each number is an 8-digit number, the first digit cannot be 0.)

\begin{sol}
    $\frac{8!}{2!3!3!} - \frac{7!}{3!2!2!} = 350$
    
\end{sol}

\question A professor has 10 identical new pens that he no longer needs. In how many ways can these pens be given to 3 students if
\begin{enumerate}[label=(\alph*)]
    \item there are no other conditions?
    \begin{sol}
        $\binom{10+3-1}{3-1} = \binom{12}{2} = 66$
    \end{sol}
    \item every student must receive at least one pen?
    \begin{sol}
        $\binom{7 + 3 - 1}{3 - 1} = \binom{9}{2} = 36$
    \end{sol}
    \item every student must receive at least two pens?
    \begin{sol}
        $\binom{4 + 3 - 1}{3 - 1} = \binom{6}{2} = 15$
    \end{sol}
    \item every student must receive at least three pens?
    \begin{sol}
        $\binom{1 + 3 - 1}{3 - 1} = \binom{3}{2} = 3$
    \end{sol}
\end{enumerate}

\question How many 4-digit numbers are there, the sum of whose digits is 11?

\begin{sol}
    $\binom{10 + 4 - 1}{4 - 1} = \binom{13}{3} = 286$ (all combinations of $(a - 1) + b + c + d = 10$)
    $286 - 3 - 1 - 3$ \\
    3 cases where $a = 10$, 1 case where $a = 11$ and 1 case each where $b,c,$ or $d = 10$
    that leaves us with $279$
\end{sol}

\question A man enters Ben's Bagels to buy a dozen bagels only to learn that even though Ben has a large supply of plain bagels and blueberry bagels, he only has two cinnamon bagels left and no other kinds.
    
\begin{enumerate}[label=(\alph*)]
    \item Express the number of selections of $r$ bagels $(r \geq 0)$ the man can make as a product of polynomials and/or power series\footnote{The book has a list of common power series in figure 9.8 on page 347.}.
    \begin{sol}
        $(1+x+x^2) \times \frac{1}{(1-x)^2}$
    \end{sol}
    \item Determine the coefficient of $x^{12}$ of the product in (a). What information does this coefficient provide?
    \begin{sol}
        we have 
        \[
            \frac{1}{1-x}
        \]
        twice (once for blueberry and once for plain) which we can write as
        \[
            \frac{1}{(1-x)^2}
        \]
        then we have 
        \[
            1+x+x^2
        \]
        for the cinnamon bagles (since we can have at most 2)
        so our combined function is 
        \[
            1+x+x^2 \times \frac{1}{(1-x)^2}
        \]
        then we go through the cases where we pick 2, 1, and 0 cinnamon bagels
        \[
            1+x+x^2 \times \frac{1}{(1-x)^2}=\sum_{n=0}^{\infty}(n+1)x^n=(1-x)^{-2}+x(1-x)^{-2}+x^2(1-x)^{-2}
        \]
        \[
            (1-x)^-2=\sum_{n=0}^{\infty}(n+1)x^n
        \]
        for $x^{12}$ we get
        \[
            13+12+11=36
        \]

        we have $36$ ways to select 12 bagels, if we can only have at most 2 cinnamon bagels.
    \end{sol}
\end{enumerate}
\question Suppose that we are interested in the number of non-negative integer solutions $a, b$ and $c$ of the equation $a + 2b + 3c = r$, where $r \in \N$.
\begin{enumerate}[label=(\alph*)]\item Determine the coefficient of $x^{12}$ of the product in (a). What information does this coefficient provide?
    \item What is this number when $r \in \{1, 2, 3, 4\}$?
    
    \begin{sol}
        $ r = 1$ has 1 solution \\
        $ r = 2$ has 2 solutions \\
        $ r = 3$ has 3 solutions \\
        $ r = 4$ has 4 solutions \\
    \end{sol}
    \item Use generating functions to describe this number for a general $r \in \N$.
    
    \begin{sol}
       \[
       a = \frac{1}{1-x}, b = \frac{1}{1-x^2}, c = \frac{1}{1-x^3}
       \]
       \[
        [x^r]\frac{1}{(1-x)(1-x^2)(1-x^3)}
        \]
    \end{sol}
\end{enumerate}
\question A man buys 6 doughnuts, each of which is a plain doughnut, a powdered doughnut or a glazed doughnut. How many possible selections are there if he buys at least one doughnut of each kind?
\begin{enumerate}[label=(\alph*)]
    \item Answer the question above by listing all possible selections in a table.
          % this is just generic table environment in latex.
          \begin{sol}
            \begin{table}[hbtp]
              \centering
              \begin{tabular}{|| c | c | c | c ||}
                  \hline
                  p & pw & g \\ \hline
                  1 & 1 & 4 \\ \hline
                  1 & 2 & 3 \\ \hline
                  1 & 3 & 2 \\ \hline
                  2 & 4 & 1 \\ \hline
                  2 & 1 & 3 \\ \hline
                  2 & 2 & 2 \\ \hline
                  2 & 3 & 1 \\ \hline
                  3 & 1 & 2 \\ \hline
                  3 & 2 & 1 \\ \hline
                  4 & 1 & 1 \\ \hline
              \end{tabular}
              \caption{couldn't make blue :'(}\label{tab:sel}
          \end{table}
          there are 10 rows, hence 10 solutions
          \end{sol}
          
    \item Answer the question above by determining the coefficient of $x^6$ in a product of polynomials and/or power series.
    \begin{sol}
        \[
            (\frac{3}{1-x})^3=\frac{x^3}{(1-x)^3}=[x^3](1-x)^{-3}
            =\sum_{n=0}^{\infty}\binom{n+2}{2}x^n
            [x^3](1-x)^{-3}=\binom{3 + 2}{2}=\binom{5}{2}=10
        \]
    \end{sol}
    \item Answer the question above by computing $\binom{s+t-1}{s}$ for an appropriate choice of $s$ and $t$.
    \begin{sol}
        \[
            \binom{3+3-1}{3}=\binom{5}{3}=10
        \]
    \end{sol}
\end{enumerate}
\end{document}
